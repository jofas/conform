\documentclass{beamer}
\usepackage[utf8]{inputenc}
\usepackage{stmaryrd}
\usepackage{tikz}
\usepackage{listings}
\usepackage{graphics}

%\usetheme{Warsaw}

\usetikzlibrary{%
    arrows,
    arrows.meta,
    decorations,
    backgrounds,
    positioning,
    fit,
    petri,
    shadows,
    datavisualization.formats.functions,
    calc,
    shapes,
    shapes.multipart,
    matrix
}

\def\version{\texttt{v0.1.0}}
\def\libconform{\texttt{libconform}}

\title{\libconform{} \version: a Python library for
       conformal prediction}

\author{Jonas Fa{\ss}bender}

\date{}

\def\setmeshr#1{
  \def\meshr{#1}
}
\def\meshr{1.25pt}

\pgfooclass{meshgrid visualizer}{
  % Stores the name of the visualizer. This is needed for
  % filtering and configuration
  \attribute name;

  % The constructor. Just setup the attribute.
  \method meshgrid visualizer(#1) { \pgfooset{name}{#1} }

  % Connect to visualize signal.
  \method default connects() {
    \pgfoothis.get handle(\me)
    \pgfkeysvalueof{/pgf/data visualization/obj}.connect(
      \me,visualize,visualize datapoint signal)
  }

  % This method is invoked for each data point. It checks
  % whether the data point belongs to the correct
  % visualizer and, if so, calls the macro \dovisualization
  % to do the actual visualization.
  \method visualize() {
    \pgfdvfilterpassedtrue
    \pgfdvnamedvisualizerfilter
    \ifpgfdvfilterpassed
      \dovisualization
    \fi
  }
}

\def\dovisualization{
  \pgfkeysvalueof{%
    /data point/\pgfoovalueof{name}/execute at begin%
  }

  \pgfpointdvdatapoint
  \pgfgetlastxy{\macrox}{\macroy}

  \pgfmathsetmacro\xlow {\macrox - \meshr}
  \pgfmathsetmacro\ylow {\macroy - \meshr}
  \pgfmathsetmacro\xhigh{\macrox + \meshr}
  \pgfmathsetmacro\yhigh{\macroy + \meshr}

  \pgfpathrectanglecorners{\pgfpoint{\xlow}{\ylow}}
                          {\pgfpoint{\xhigh}{\yhigh}}

  \pgfkeysvalueof{%
    /data point/\pgfoovalueof{name}/execute at end%
  }
}
\tikzdatavisualizationset{
  visualize as meshgrid/.style={
    new object={
      when=after survey,
      store=/tikz/data visualization/visualizers/#1,
      class=meshgrid visualizer,
      arg1=#1
    },
    new visualizer={#1}{%
      color=visualizer color,
      every path/.style={fill,draw,opacity=0.5},
    }{},
    /data point/set=#1
  },
  visualize as meshgrid/.default=meshgrid
}

\begin{document}
\maketitle

\section{Conformal prediction}

\begin{frame}[fragile]
  \frametitle{Conformal prediction}
  \setbeamercovered{invisible}
  \pause
  \begin{tikzpicture}[scale=0.75]
    \datavisualization [
      scientific axes={clean, upright labels},
      visualize as scatter/.list={b,r},
      x axis={label={$\textbf{X}_1$}},
      y axis={label={$\textbf{X}_2$}},
      b={style={mark=*, visualizer color=blue}},
      r={style={mark=*, visualizer color=red}},
    ]
      data[ headline={x, y}, set=b
          , read from file={blue_points.csv}]
      data[ headline={x, y}, set=r
          , read from file={red_points.csv}]
      ;
  \end{tikzpicture}

  \begin{tikzpicture}[remember picture, overlay]
    \node[color=blue, circle, draw,fill,
      label=right:\textcolor{blue}{$label_1$},
      inner sep=2pt] at (6,3) (l_zero) {};
    \node[color=red, circle,draw,fill,
      label=right:\textcolor{red}{$label_2$},
      below=0.2 of l_zero,
      inner sep=2pt] (l_one) {};
  \end{tikzpicture}
  \pause

  \begin{itemize}[<+->]
    \item Feature space $\textbf{X} := \mathbb{R}^2$
    \item Label space $\textbf{Y} := \{
          \textcolor{blue}{label_1},
          \textcolor{red}{label_2}\}$
    \item Example space $\textbf{Z} := \textbf{X} \times
          \textbf{Y}$
    \item Example $z_i := (x_i, y_i); x_i \in \textbf{X},
          y_i \in \textbf{Y}$
    \item Datensatz $\Lbag z_1,\dots,z_n \Rbag$
  \end{itemize}
\end{frame}

\begin{frame}[fragile]
  \frametitle{Conformal prediction}
  \setbeamercovered{invisible}
  \pause
  \begin{tikzpicture}[scale=0.75]

    \datavisualization [
        scientific axes={clean, upright labels},
        visualize as line/.list={d},
        visualize as scatter/.list={b,r},
        x axis={label={$\textbf{X}_1$}},
        y axis={label={$\textbf{X}_2$}},
        b={style={mark=*, visualizer color=blue}},
        r={style={mark=*, visualizer color=red}},
      ]
        data[set=d] {
          x, y
          0.0, 0.0
          1.0, 1.0
        }
        data[ headline={x, y}, set=b
            , read from file={blue_points.csv}]
        data[ headline={x, y}, set=r
            , read from file={red_points.csv}]
        ;
  \end{tikzpicture}

  \begin{tikzpicture}[remember picture, overlay]
    \node[color=blue, circle, draw,fill,
      label=right:\textcolor{blue}{$label_1$},
      inner sep=2pt] at (6,3) (l_zero) {};
    \node[color=red, circle,draw,fill,
      label=right:\textcolor{red}{$label_2$},
      below=0.2 of l_zero,
      inner sep=2pt] (l_one) {};

    \draw ($(l_one) - (-0.12,0.35)$) --
          ($(l_one) - (0.12,0.35)$);

    \node[below=0.2 of l_one, inner sep=2pt,
      label=right:\textcolor{black}{%
        $D_{\Lbag z_1,\dots,z_n \Rbag}$}] {};
  \end{tikzpicture}

  \begin{itemize}[<+->]
    \item Klassischer Machine Learning Predictor
      $D_{\Lbag z_1,\dots,z_n \Rbag}$

    \item Bare predictions, kein confidence Wert in
          prediction

    \item Kann zu sog. nonconformity measure umgewandelt
          werden (Basis von CP)
  \end{itemize}
\end{frame}

\begin{frame}[fragile]
  \frametitle{Conformal prediction}
  \setbeamercovered{invisible}
  \pause

  \begin{tikzpicture}[scale=0.75]
    \datavisualization [
      scientific axes={clean, upright labels},
      visualize as scatter/.list={b,r},
      visualize as meshgrid/.list={%
        meshb0, meshb1, meshb2, meshb3,
        meshr0, meshr1, meshr2, meshr3
      },
      x axis={label={$\textbf{X}_1$}},
      y axis={label={$\textbf{X}_2$}},
      b={style={mark=*, visualizer color=blue}},
      r={style={mark=*, visualizer color=red}},
      meshb0={style={color=blue!60}},
      meshb1={style={color=blue!45}},
      meshb2={style={color=blue!30}},
      meshb3={style={color=blue!15}},
      meshr0={style={color=red!60}},
      meshr1={style={color=red!45}},
      meshr2={style={color=red!30}},
      meshr3={style={color=red!15}},
    ]
    data[ headline={x, y}, set=b
        , read from file={blue_points.csv}]
    data[ headline={x, y}, set=r
        , read from file={red_points.csv}]
    data[
      headline={x, y},
      read from file={data/0.02_0.csv},
      set=meshb0
    ]
    data[
      headline={x, y},
      read from file={data/0.03_0.csv},
      set=meshb1
    ]
    data[
      headline={x, y},
      read from file={data/0.04_0.csv},
      set=meshb2
    ]
    data[
      headline={x, y},
      read from file={data/0.05_0.csv},
      set=meshb3
    ]
    data[
      headline={x, y},
      read from file={data/0.02_1.csv},
      set=meshr0
    ]
    data[
      headline={x, y},
      read from file={data/0.03_1.csv},
      set=meshr1
    ]
    data[
      headline={x, y},
      read from file={data/0.04_1.csv},
      set=meshr2
    ]
    data[
      headline={x, y},
      read from file={data/0.05_1.csv},
      set=meshr3
    ]
    ;
  \end{tikzpicture}

  \begin{tikzpicture}[remember picture, overlay]
    \begin{scope}[label distance=0.6cm]
      \node[color=blue, circle, draw,fill,
        label=right:\textcolor{blue}{$label_1$},
        inner sep=1.5pt] at (6,3.75) (l_zero) {};
      \node[color=red, circle,draw,fill,
        label=right:\textcolor{red}{$label_2$},
        below=0.2 of l_zero,
        inner sep=1.5pt] (l_one) {};
    \end{scope}

    \node[color=blue!60,opacity=0.5,
      below=0.2 of l_one, fill, draw] (m_zero) {};
    \node[color=red!60,opacity=0.5,right=0.2 of m_zero,
      label=right:\textcolor{black}{$\epsilon_1 := 0.2$},
      fill, draw] {};

    \node[color=blue!45,opacity=0.5,
      below=0.2 of m_zero, fill, draw] (m_one) {};
    \node[color=red!45,opacity=0.5,right=0.2 of m_one,
      label=right:\textcolor{black}{$\epsilon_2 := 0.3$},
      fill, draw] {};

    \node[color=blue!30,opacity=0.5,
      below=0.2 of m_one, fill, draw] (m_two) {};
    \node[color=red!30,opacity=0.5,right=0.2 of m_two,
      label=right:\textcolor{black}{$\epsilon_3 := 0.4$},
      fill, draw] {};

    \node[color=blue!15,opacity=0.5,
      below=0.2 of m_two, fill, draw] (m_three) {};
    \node[color=red!15,opacity=0.5,right=0.2 of m_three,
      label=right:\textcolor{black}{$\epsilon_4 := 0.5$},
      fill, draw] {};
  \end{tikzpicture}

  \begin{itemize}[<+->]
    \item Conformal Predictor
      $\Gamma^{\epsilon}_{\Lbag z_1,\dots,z_n \Rbag}$

    \item Wichtigste Eigenschaft: validity under
          exchangeability

    \item $\Gamma^{\epsilon}_{\Lbag z_1,\dots,z_n \Rbag}$
          hat Genauigkeit von mindestens $1 - \epsilon$
          (wenn $z_1,\dots,z_n$ exchangeable)

    \item In Realit\"at: wahre exchangeablility selten,
          aber meistens nah genug dran
          %(Abhängig von
          %Datenqualit\"at, Datensatzgr\"o{\ss}e und Feature
          %space gr\"o{\ss}e)
  \end{itemize}
\end{frame}

\section{\libconform}

\begin{frame}[fragile]
  \frametitle{\libconform}
  \setbeamercovered{invisible}
  \pause

  \lstset{%
    basicstyle=\footnotesize,
    breaklines = true,
    keywordstyle=\bfseries\color{green!70!black},
    basicstyle=\ttfamily\color{black},
    commentstyle=\itshape\color{purple},
    identifierstyle=\color{blue},
    stringstyle=\color{orange},
    showstringspaces=false,
    rulecolor=\color{black},
  }
  \scalebox{0.5}{%
    \lstinputlisting[language=Python]{example.py}
  }
\end{frame}

\begin{frame}
  \frametitle{\libconform}
  \setbeamercovered{invisible}
  \pause
  \begin{itemize}[<+->]
    \item Python: lingua franca f\"ur Machine Learning

    \item MIT-licensed

    \item \textbf{Unstable}

    \item Fokus: extensibility

    \item Grundlegensten Algorithmen der CP-Familie
          implementiert (CP, smoothed CP, inductive CP,
          mondrian CP, RRCM, Venn prediction,\dots)

  \end{itemize}
\end{frame}

\begin{frame}
  \frametitle{\libconform{} -- TODO}
  \setbeamercovered{invisible}
  \pause
  \begin{itemize}[<+->]
    \item Test-dichte zu gering

    \item Unausgereifte interne APIs

    \item Single-threaded und langsam

    \item Weitere Algorithmen der CP-Familie implementieren
          (aggregated CP cross-conformal prediction,
          Venn-Abers,\dots)

    \item Mehr nonconformity scores

  \end{itemize}
\end{frame}

\begin{frame}
  \frametitle{Vielen Dank}
  Bei Interesse:
  \begin{itemize}
    \item \url{https://github.com/jofas/conform/}
    \item jonas@fassbender.dev
  \end{itemize}
\end{frame}

\end{document}
