\documentclass[twoside,11pt]{article}
\usepackage{jmlr2e}
\usepackage{fontawesome}
\usepackage[page]{appendix}
\usepackage{xcolor}
\usepackage[marginparsep=30pt]{geometry}
\usepackage{marginnote}

\def\version{\texttt{v0.1.0}}
\def\libconform{\texttt{libconform}}
\def\info{\textcolor{blue}{\faInfo}}
\def\alert{\textcolor{red}{\faExclamation}}

\title{\libconform{} \version: a Python library for
       conformal prediction}

\author{\name Jonas Fa{\ss}bender
        \email jonas@fassbender.dev}

\ShortHeadings{\libconform{} \version{}}{Fa{\ss}bender}
\firstpageno{1}

\begin{document}

\maketitle

%\begin{abstract}%
%\end{abstract}

%\begin{keywords}
%\end{keywords}


\section{Introduction}

This paper introduces the Python library \libconform,
implementing concepts defined in \citet{alrw}, namely the
conformal prediction framework and Venn prediction for
reliable machine learning.
These algorithms address a weakness of more traditional
machine learning algorithms which produce only bare
predictions, without their confidence in them/the
probability of the prediction, therefore providing no
measure of likelihood, desirable and even necessary in many
real-world application domains.

The conformal prediction framework is composed of
variations of the conformal prediction algorithm (CP),
first described in
\citet{vovk_et_al_1999, saunders_et_al_1999}.
A conformal predictor provides a measurement of confidence
in its predictions.
A Venn predictor, on the other hand, provides a
multi-probabilistic measurement, making it a probabilistic
predictor.
Below in the text, Venn predictors are included if only
``conformal prediction framework'' is written, except
stated otherwise.

The conformal prediction framework is applied successfully
in many real-world domains, for example face recognition,
medical diagnostic and prognostic and network traffic
classification \citep[see][part 3]{cprml}.

It is build on traditional machine learning algorithms, the
so called underlying algorithms
\citep[see][]{papadopoulos_et_al_2007}, which makes Python
the first choice for implementation, since its machine
learning libraries are top of the class, still evolving and
improving due to the commitment of a great community of
developers and researchers.

\libconform's aim is to provide an easy to use, but very
extensible API for the conformal prediction framework, so
developers can use their preferred implementations for the
underlying algorithm and can leverage the library, even in
this early stage.
\libconform{} \version{} is \textbf{not} yet stable; there
are still features missing and the API is very likely to
change and improve.
The library is licensed under the MIT-license and its
source code can be downloaded from
\url{https://github.com/jofas/conform}.

This paper combines \libconform{}'s documentation with a
short introduction to the implemented algorithms.
Paragraphs marked with \info{} contain general information
about the library and descriptions of the
internal workings, while paragraphs marked
with \alert{} describe changes in future versions.

Appendix~\ref{appendix:a} provides an overview over
\libconform's API and Appendix~\ref{appendix:b} contains
examples on how to use the library.

%\marginnote{\info{}}[2.7em]

\section{Nonconformity scores}

\section{Conformal predictors}

\subsection{Smoothed conformal predictors}

\subsection{Conformal predictor for regression: ridge
            regression confidence machine}

\section{Inductive conformal predictors}

\section{Mondrian (inductive) conformal predictors}

\section{Probabilistic prediction: Venn predictors}

\section{Meta-conformal predictors}

\section{Conclusion}

\renewcommand{\appendixpagename}{}
\begin{appendices}
  \section*{Appendices}

  \section{API reference}
  \label{appendix:a}

  \section{Examples}
  \label{appendix:b}

\end{appendices}

\bibliography{libconform.bib}

\end{document}
