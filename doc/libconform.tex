\documentclass[twoside,11pt]{article}
\usepackage{jmlr2e}
\usepackage{amsmath}
\usepackage[page]{appendix}
\usepackage{xcolor}
\usepackage[marginparsep=30pt]{geometry}
\usepackage{marginnote}
\usepackage{stmaryrd}
\usepackage{algorithm}
\usepackage{algorithmic}

\def\version{\texttt{v0.1.0}}
\def\libconform{\texttt{libconform}}

\title{\libconform{} \version: a Python library for
       conformal prediction}

\author{\name Jonas Fa{\ss}bender
        \email jonas@fassbender.dev}

\ShortHeadings{\libconform{} \version{}}{Fa{\ss}bender}
\firstpageno{1}

\begin{document}

\maketitle

%\begin{abstract}%
%\end{abstract}

%\begin{keywords}
%\end{keywords}


\section{Introduction}

This paper introduces the Python library \libconform,
implementing concepts defined in \citet{alrw}, namely the
conformal prediction framework and Venn prediction for
reliable machine learning.
These algorithms address a weakness of more traditional
machine learning algorithms which produce only bare
predictions, without their confidence in them/the
probability of the prediction, therefore providing no
measure of likelihood, desirable and even necessary in many
real-world application domains.

The conformal prediction framework is composed of
variations of the conformal prediction algorithm (CP),
first described in
\citet{vovk_et_al_1999, saunders_et_al_1999}.
A conformal predictor provides a measurement of confidence
in its predictions.
A Venn predictor, on the other hand, provides a
multi-probabilistic measurement, making it a probabilistic
predictor.
Below in the text, Venn predictors are included if only
``conformal prediction framework'' is written, except
stated otherwise.

The conformal prediction framework is applied successfully
in many real-world domains, for example face recognition,
medical diagnostic and prognostic and network traffic
classification \citep[see][part 3]{cprml}.

It is build on traditional machine learning algorithms, the
so called underlying algorithms
\citep[see][]{papadopoulos_et_al_2007}, which makes Python
the first choice for implementation, since its machine
learning libraries are top of the class, still evolving and
improving due to the commitment of a great community of
developers and researchers.

\libconform's aim is to provide an easy to use, but very
extensible API for the conformal prediction framework, so
developers can use their preferred implementations for the
underlying algorithm and can leverage the library, even in
this early stage.
\libconform{} \version{} is \textbf{not} yet stable; there
are still features missing and the API is very likely to
change and improve.
The library is licensed under the MIT-license and its
source code can be downloaded from
\url{https://github.com/jofas/conform}.

This paper combines \libconform{}'s documentation with a
outline of the implemented algorithms.
At the end of each chapter there are notes on the
implementation containing general information about the
library, descriptions of the internal workings and the API
and possible changes in future versions.

Appendix~\ref{appendix:a} provides an overview over
\libconform's API and Appendix~\ref{appendix:b} contains
examples on how to use the library.

\section{Conformal predictors}

Like stated in the introduction, this chapter will only
outline conformal prediction (CP). For more details see
\citet{alrw}.

CP---like the name suggests---determines the label(s) of an
incoming observation based on how well it/they conform(s)
with previous observed examples. Conformal prediction can
be used either in the online or the offline, or batch
learning setting.
The offline learning setting, compared to online learning,
weakens the validity---described below in this chapter---of
the classifier in favor of computational efficiency
\citep[see][Chapter 3]{alrw}.

Let $\Lbag z_1,\dots,z_n \Rbag$ be a bag, also called
multiset%
\footnote{It is typical in machine learning to denote this
as a data set, even though examples do not have to
be unique, making the so called set a multiset. A multiset
is not a list, since the ordering of the elements is not
important.},
of examples, where each example $z_i \in \textbf{Z}$ is a
tuple $(x_i,y_i); x_i \in \textbf{X}, y_i \in \textbf{Y}$.
\textbf{X} is called the observation space and \textbf{Y}
the label space. For this time \textbf{Y} is considered
finite, making the task of prediction a classification
task, rather than regression, which will be considered in
Chapter~\ref{subsec:rrcm}.

A conformal predictor can be defined as a confidence
predictor $\Gamma$. For this an input $\epsilon \in (0,1)$,
the significance level is needed. $1 - \epsilon$ is called
the confidence level.
A conformal predictor $\Gamma^\epsilon$ in the online
setting is conservatively valid under the exchangeability
assumption, which means, as long as exchangeability holds,
it makes errors at a frequency of $\epsilon$ or less. For
more on that refer to \citet[][Chapters 1,2,7]{alrw}.

CP, in its original setting, produces nested prediction
sets. Rather than returning a single label as its
prediction, it returns a set of elements
$\textbf{Y}^\prime \in 2^{\textbf{Y}}$, $2^{\textbf{Y}}$
being the set of all subsets of $\textbf{Y}$, including the
empty set.
The prediction sets are called nested, because, for
$\epsilon_1 \geq \epsilon_2$, the prediction of
$\Gamma^{\epsilon_1}$ is a subset of $\Gamma^{\epsilon_2}$
\citep[see][Chapter 2]{alrw}.

In order to predict the label of a new observation
$x_{n+1}$, set $z_{n+1}:=(x_{n+1}, y)$, for each
$y \in \textbf{Y}$ and check how $z_{n+1}$ conforms with
the examples of our bag $\Lbag z_1,\dots,z_n \Rbag$.

This is done with a nonconformity measure
$A_{n+1}:\textbf{Z}^n \times \textbf{Z} \rightarrow
\mathbb{R}$. First, $z_{n+1}$ is added to the bag, then
$A_{n+1}$ assigns a numerical score to each example in
$z_i$:
\begin{align}
  \alpha_i = A_{n+1}(\Lbag z_1,\dots,z_{i-1},z_{i+1},
             \dots,z_{n+1} \Rbag, z_i).
\label{eq:a0}
\end{align}
One can see in this equation that $z_i$ is removed from the
bag. It is also possible to compute $\alpha_i$ with $z_i$
in the bag, which means for
$A_{n+1}:\textbf{Z}^{n+1} \times \textbf{Z} \rightarrow
\mathbb{R}$ the score is computed as:
\begin{align}
  \alpha_i = A_{n+1}(\Lbag z_1,\dots,z_{n+1} \Rbag, z_i).
\label{eq:a1}
\end{align}
Which one is preferable is case-dependent
\citep[see][Chapter 4.2.2]{shafer_et_al_2008}.

$\alpha_i$ is called nonconformity score.
The nonconformity score can now be used to compute the
p-value for $z_{n+1}$, which is the fraction of examples
from the bag which are at least as nonconforming as
$z_{n+1}$:
\begin{align}
  \frac{|\{i=1,\dots,n+1: \alpha_i \geq \alpha_{n+1}\}|}
       {n + 1}.
\label{eq:p0}
\end{align}
Another way to determine the p-value is through smoothing,
in which case the nonconformity scores equal to
$\alpha_{n+1}$ are multiplied by a random value
$\tau_{n+1}$:
\begin{align}
  \frac{|\{i=1,\dots,n+1: \alpha_i > \alpha_{n+1}\}|
    + \tau_{n+1} |\{i=1,\dots,n+1:\alpha_i=\alpha_{n+1}\}|}
       {n + 1}
\label{eq:p1}
\end{align}
A conformal predictor using the smoothed p-value is called
a smoothed conformal predictor and is exactly valid under
exchangeability in the online setting, which means it makes
errors at a rate exactly $\epsilon$
\citep[see][Chapter 2]{alrw}.
If the p-value of $z_{n+1}$ is larger than $\epsilon$, $y$
is added to the prediction set.

\begin{algorithm}
  \caption{: Conformal predictor $\Gamma^\epsilon
    (\Lbag z_1,\dots,z_n \Rbag, x_{n+1})$}
  \label{alg:cp}

  \begin{algorithmic}[1]
    \FORALL{$y \in \textbf{Y}$}
      \STATE{set $z_{n+1} := (x_{n+1}, y)$
             and add it to the bag}
      \FORALL{$i=1,\dots,n+1$}
        \STATE{compute $\alpha_i$ with (\ref{eq:a0})
               or (\ref{eq:a1})}
      \ENDFOR
      \STATE{set $p_y$ with (\ref{eq:p0}) or (\ref{eq:p1})}
      \IF{$p_y > \epsilon$}
        \STATE{add $p_y$ to prediction set}
      \ENDIF
    \ENDFOR
    \RETURN{prediction set}
  \end{algorithmic}
\end{algorithm}

\noindent
\textbf{Notes on the implementation:}
\libconform{} provides the \texttt{CP} class for creating
conformal prediction classifiers. \libconform{}'s
classifier classes provide quite equal APIs, only with
minor variations.
The API of the classifier classes is comparable to major
machine learning libraries like sklearn or keras
\citep[see][]{sklearn_api, keras}.

It is common in machine learning to split the learning task
in two distinct operations, first a
training---or fit---operation on a training set of examples
and then a predict operation on new observations.
\libconform{}'s classifier classes follow this style,
providing a \texttt{train} and a \texttt{predict} method.

While this split in training and predicting is common for
inductive classifiers, which first derive a prediction
rule, or decision surface, from the training set and then
predict unseen examples inductively based on that rule,
it is not really the way CP works. CP was designed to be
transductive, not inductive, which means rather than
to generate a prediction rule, it uses all previous seen
examples to classify a new observation, making the training
step unnecessary (see Algorithm~\ref{alg:cp}).
While the transductive setting is more elegant than the
inductive setting, it is computationally very expensive and
not feasible for larger data sets and for underlying
algorithms---discussed in Chapter~\ref{subsec:ncs}---which
have a computationally complex training phase
\citep[see][Chapter 1]{papadopoulos_et_al_2007,alrw}.

\libconform{}'s aim is to be---one day---ready for
production, where, for some application domains, the time
complexity of predicting a new observation is crucial.
Therefore \libconform{}'s \texttt{CP} class tries to
minimize the time complexity of its \texttt{predict}
method. Instead of adding $z_{n+1}$ to the bag and then
computing $\alpha_i$ for each example in the bag during
prediction, it computes $\alpha_1,\dots,\alpha_n$ during
training and only computes $\alpha_{n+1}$ in
\texttt{predict} (see Algorithm~\ref{alg:cp}, lines 3-5).

Arguably \texttt{CP} does not implement the conformal
prediction algorithm in its original form, being currently
rather a special case of inductive conformal prediction,
where the calibration set is equal to the whole bag of
examples previously witnessed, instead of a subset (see
Chapter~\ref{sec:icp}).
It is possible that \texttt{CP} will change to being the
implementation of the original conformal prediction
algorithm described in \citet[][Chapter 2]{alrw} in a
future version.

\texttt{CP} takes an instance of a nonconformity measure
$A$ and a sequence of $\epsilon_1,\dots,\epsilon_{g}$
as its arguments during initialization, therefore being the
implementation of
$\Gamma^{\epsilon_1},\dots,\Gamma^{\epsilon_g}$.

It also provides to extra utility methods for validation,
\texttt{score} and \texttt{score\_online}, which generate
metrics for the conformal predictors
$\Gamma^{\epsilon_1},\dots,\Gamma^{\epsilon_g}$.
The most important of those metrics are the error rates
$Err_1,\dots,Err_{g}$.
It the error rate $Err_i \leq \epsilon_i$ over the bag of
examples provided to \texttt{score}/\texttt{score\_online}
than $\Gamma^{\epsilon_i}$ was valid on the bag.

\texttt{score\_online} adds an example, after it was
predicted, to the training bag and calls \texttt{train},
using $\Gamma^{\epsilon_1},\dots,\Gamma^{\epsilon_g}$ in
the online learning setting.

\subsection{Nonconformity measures based on underlying
            algorithms}
\label{subsec:ncs}

% D(x)

% here info on how NCScores impl in libconform

\subsection{Conformal predictor for regression: ridge
            regression confidence machine}
\label{subsec:rrcm}

\section{Inductive conformal predictors}
\label{sec:icp}

\section{Mondrian (inductive) conformal predictors}

\section{Probabilistic prediction: Venn predictors}

\section{Meta-conformal predictors}

\section{Conclusion}

\renewcommand{\appendixpagename}{}
\begin{appendices}
  \section*{Appendices}

  \section{API reference}
  \label{appendix:a}

  \section{Examples}
  \label{appendix:b}

\end{appendices}

\bibliography{libconform.bib}

\end{document}
